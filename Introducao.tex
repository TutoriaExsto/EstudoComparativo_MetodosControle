\section{Introdução}
    Um processo industrial caracteriza uma série de operações de ordem mecânica, física ou química cuja função é a fabricação de algum bem ou material e define etapas implementadas para que a produção desse item seja a mais otimizada possível \cite{Definicao_Processo}. Um processo bem estabelecido é fundamental para que a cadeia envolvida na fabricação seja realizada de forma correta. A segurança da execução das operações é dada pelo monitoramento e controle das variáveis do processo, ou seja, da medida de grandezas físicas envolvidas nas etapas.
    
    Entre as variáveis de processo mais monitoradas, destaca-se a temperatura, cujo controle, muitas vezes, é um fator primordial para o desenvolvimento das etapas de fabricação. A medição incorreta dessa grandeza pode ocasionar danos irreversíveis, afetando, negativamente, a eficiência e consumo de energia do processo. Por conseguinte, o controle correto e uma medida precisa dessa grandeza são de suma importância para a manufatura \cite{Apostila_Sense}.
    
    Os sistemas para controle de temperatura são apresentados em diversas etapas do processo como, por exemplo, na transformação de matéria prima em indústrias metalúrgicas, cozimento e preparação em indústrias alimentícias, ou conservação de medicamentos na indústria farmacêutica \cite{Apostila_Sense}. A implementação, nessas etapas, pode trazer diversos benefícios em relação à qualidade do processo e economia no tempo de produção, tais como a melhoria da qualidade e quantidade do produto e a redução de desperdício de material.
    
    Além do controle, o monitoramento dessa variável torna-se evidente em processos de atuação em zonas classificadas, em que os equipamentos do processo possuem o conceito de Segurança Intrínseca (Ex i), ou seja, o controle de temperatura atua para evitar a condição exotérmica, situação que pode levar à falha geral e explosão do processo \cite{Seguranca_Intrinseca}.
    
    Os sistemas responsáveis pelo controle da temperatura trabalham referenciados ao erro da grandeza no processo, valor definido pela diferença entre a temperatura desejada e a temperatura entregue pelo elemento gerador de calor. Tais dispositivos, considerando uma planta industrial, apresentam-se, majoritariamente, na forma de fornos ou caldeiras.
    
    Enfatizando a indústria alimentícia, destacam-se, como elementos geradores de calor, os fornos de convecção forçada. Este modelo de forno trabalha movimentando um fluído, normalmente o ar, artificialmente através de uma bomba, ventilador ou dispositivo similar, a fim de aumentar a circulação do fluído na câmara de cocção. A utilização desse princípio de funcionamento, nos fornos de convecção forçada, reduz de $25\%$ a $75\%$ a temperatura ajustada quando comparado a um forno de convecção natural \cite{ControlesTipicos}, além de otimizar a realização de alguns processos nesse setor. Um controle ótimo de temperatura para esse dispositivo também minimiza seu consumo de energia, tornando um elemento eficiente e essencial no dinamismo desse tipo de manufatura.
    
    Dadas as características do funcionamento desse tipo de forno, bem como a importância e as vantagens do controle correto da temperatura, esse trabalho busca projetar sistemas de controle de temperatura para um forno de convecção forçada, baseados em diferentes técnicas, como, por exemplo, Controle \textit{On/Off}, Controlador Proporcional-Integral-Derivativo (PID) e por Lógica Difusa (\Fuzzy), com o intuito de comparação entre os desempenhos em relação à estabilidade do sistema, velocidade da resposta e consumo de energia no processo de cozimento nas indústrias alimentícias.
